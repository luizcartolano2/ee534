\documentclass{article}

\usepackage[brazil]{babel}
\usepackage[utf8]{inputenc}
\usepackage[T1]{fontenc}% optional T1 font encoding
\usepackage[%
    colorlinks=true,
    pdfborder={0 0 0},
    linkcolor=red
]{hyperref}
\usepackage[all]{hypcap}
\usepackage{amsmath}
\interdisplaylinepenalty=2500
\usepackage{graphicx}
\usepackage[cmintegrals]{newtxmath}
\usepackage{cite}
\usepackage{listings}
\usepackage{hyperref}
\usepackage{indentfirst}
\usepackage{siunitx}
\usepackage{textgreek}
\usepackage[portuguese,linesnumbered,ruled]{algorithm2e}
\usepackage{multirow}
\usepackage{anysize}

\begin{document}

\title{Preparação do Exp. III — Pré amplificador de áudio com Transistores de Efeito de Campo (FET)}
\author{Bianca Yoshie Itiroko - 164923, Luiz Eduardo Cartolano - 183012, Seong Eun Kim - 177143 \\ EE534 - Turma Y - Grupo 2}
\date{Setembro de 2018}

\maketitle

\section{Caracterização do FET} \label{sec:1}
A partir do datasheet\cite{ref:datasheet} do transistor foi possível encontrar que V$_{TH}$ é igual a 0,8V. Usando a equação dada:
\begin{equation}
    k = \frac{I_{D,On}}{(V_{GS,On} - V_{TH})^2}
    \label{eq:k}
\end{equation}
Temos que para V$_{GS}$ = $1,4V$, e I$_{D,On}$ = $1mA$, o valor de k é 0,0027 A/V$^2$.

\section{Projeto de um Pré-Amplificador}
Considerando os parâmetros dados pelo enunciado \cite{ref:roteiro} e os valores do item \ref{sec:1}, primeiro é preciso encontrar o valor de I$_D$:
\begin{equation}
    I_D = \frac{1}{2} * K * (V_{GS} - V_{DS})^2 = 1mA
\end{equation}

Sendo V$_{DS}$ dado por:
\begin{equation}
    V_{DS} = V_{CC} - R_D * I_D    
\end{equation}
\newline

Então R$_D$ é igual a:
\begin{equation}
    R_D = \frac{-A_V}{K*\frac{-V_{CC}}{A_C - 1}}
\end{equation}
\newline

De onde podemos concluir que R$_D$ = 5$K\Omega$.
\newline

Sendo V$_{GS}$ dado por:
\begin{equation}
    V_{GS} = \frac{V_{CC}}{R_1 + R_2} * R_2
\end{equation}

E R$_{in}$ igual a:
\begin{equation}
    R_{in} = \frac{R_1 * R_2}{R_1 + R_2}
\end{equation}

Então temos que:
\begin{equation}
    500 * 10^3 = \frac{v_{GS}}{V_{CC}} * R_1
\end{equation}
\begin{equation}
    R_1 = 500*10^3 * \frac{12}{1,4}
\end{equation}

Obtendo que R$_1 \approx 4M\Omega$.
\newline

Uma vez que sabemos o valor de R$_1$, é possível então encontrar R$_2$, fazendo:
\begin{equation}
    \frac{1}{R_{in}} = \frac{1}{R_1} + \frac{1}{R_2}
\end{equation}
\begin{equation}
    \frac{1}{500} = \frac{1}{4000} + \frac{1}{R_2}
\end{equation}

Obtendo que R$_2 \approx 571,42k\Omega$.
\newline

A simulação do circuito pode ser acessada em \cite{ref:simu}.
\newline

Para calcular a potência dissipada em cada resistor, utilizamos a fórmula:
\begin{equation}
    P = \frac{V^2}{R}
\end{equation}

Para R$_1$,
\begin{equation}
    V_{R_1} = \frac{R_1}{R_1 + R_2} * V_{cc} = 10,5V
\end{equation}

\begin{equation}
    P_{R_1} = \frac{V_{R_1}^2}{R_1} = 2,75 * 10^{-5}W
\end{equation}

Para R$_2$,
\begin{equation}
    V_{GS} = \frac{R_2}{R_1 + R_2} * V_{cc} = 1,5V
\end{equation}

\begin{equation}
    P_{R_2} = \frac{V_{GS}^2}{R_2} = 3,9*10^{-6}W
\end{equation}

Para R$_d$,
\begin{equation}
    P_{R_d} = R_d * I_d^2 = 5*10^{-3}W
\end{equation}
\newline

Por fim, podemos dizer que o transistor \emph{BSS100} pode ser utilizado 
para V$_{in}$ = 10$mVp$, pois substituindo os valores encontrados para as resistências, e usando a tensão em questão no circuito simulado, é possível verificar seu funcionamento.



\nocite{*}
\bibliographystyle{plain}
\bibliography{references}

\end{document}

