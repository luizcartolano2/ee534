 \section{Procedimentos} \label{sec:procedimentos}
    Para a realização dos experimentos propostos, foram utilizados os seguintes componentes e ferramentas: Osciloscópio digital de dois canais, gerador de ondas/funções, fonte de tensão contínua, cabos com plugs banana e coaxial, multímetro digital, placa de contatos, transístores \emph{BSS100 ,2N2222 e 2N2907}, capacitores de $220nF$ e $680nF$, resistores de $10 k\Omega$ e $100 k\Omega$, diodos \emph{1N4004}, um amplificador operacional $LM324$ e dois potenciômetros um linear de $10 k\Omega$ e um logarítmico de $50 k\Omega$. Além de um alto-falante e um microfone.
    
    Na primeira parte buscou-se observar a linearidade do circuito da Figura \ref{fig:circ1} e mediu-se a distorção harmônica do sinal de saída, como será melhor explicado na Seção \ref{sec:discussao}.
    
    Em um segundo momento, o foco foi dado no estudo de implementações básicas do amplificador operacional. Primeiro, montou-se o circuito da Figura \ref{fig:circ2}, para tal usou-se o potenciômetro linear de $10k\Omega$. O objetivo desta etapa foi entender o funcionamento do potenciômetro e também observar as saturações do amplificador. Depois de familiarizado com esse circuito, montou-se o observado na Figura \ref{fig:circ3}, que é um pouco mais complexo, e que introduz o conceito da \emph{realimentação} (ou \emph{feedback}). Neste momento, o principal foco foi encontrar o ponto de máxima excursão do amplificador montado, como explicaremos na Seção \ref{sec:discussao}. Um detalhe importante neste momento do laboratório é entender a estrutura do amplificador operacional, como pode ser visto na Figura \ref{fig:circAmpOp}, existem quatro \emph{amp ops} no componente usado, logo, é importante se atentar as portas usadas. Além disso, o mais importante, é que o componente usado necessita estar polarizado para funcionar, por isso é essencial conectar uma fonte de tensão constante na porta 4 e uma tensão (oposta ou conectar ao \emph{ground}) na porta 11.
    
    Uma vez acostumado com o \emph{amp op} acoplou-se o amplificador do parágrafo anterior ao \emph{push pull}, obtendo o circuito da Figura \ref{fig:circ4}. Com o circuito montado, buscou-se, primeiro, obter o ponto no qual o sinal de saída não estivesse distorcido. E então, variou-se a frequência da tensão de entrada, desde $1Hz$ até $1 MHz$, visando observar o comportamento do sinal de saída.
    
    Por fim, acoplou-se ao circuito um estágio de ganho, obtendo o circuito da Figura \ref{fig:circ5}, para este, conectou-se um microfone e "brincou-se" com o som obtido a partir de diferentes entradas sonoras.
    