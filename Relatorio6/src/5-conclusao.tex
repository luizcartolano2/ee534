 \section{Conclusão} \label{sec:conclusao}
    Neste experimento, buscou-se analisar e entender como utilizar o estágio de potência tipo push-pull como componente de nosso circuito, mas sem comprometer o sinal com grades distorções.
    Ao adicionar um amplificador operacional e um potenciômetro ao circuito push pull montado na aula anterior, verificou-se que o circuito pode ser melhor aproveitado, uma vez que é possível encontrar uma tensão no potenciômetro que acrescente um offset que seja totalmente aproveitado.
    Devido aos problemas técnicos enfrentados por conta da queima do transistor NPN, não foi possível avaliar o quão mais preciso e com menos distorções a adição dos novos elementos proporcionou ao circuito.