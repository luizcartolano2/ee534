\section{Introdução} \label{sec:introducao}
    Nos últimos laboratórios, estudou-se a importância dos transistores \emph{BJT} e sua aplicação nos circuitos \emph{push-pull}, neste experimento será estudada uma variação deste circuito que faz uso de outro componente super importante no mundo da eletrônica, os \emph{amplificadores operacionais}.
    
    Os amplificadores operacionais (ou \emph{amp ops}, como são comumente conhecidos), é um amplificador com ganho muito alto, impedância de entrada alta e impedância de saída baixa. Como podemos ver na Figura \ref{fig:ampOp}, ele possui duas entradas (um terminal \emph{inversor(-)} e um \emph{não-inversor(+)}) além de um terminal de saída. A tensão de saída é dada pela diferença entre os terminais multiplicada pelo ganho em malha aberta, como descrito pela Equação \ref{eq:ganhoAmpOp}.
    
    Suas principais aplicações, como o próprio nome diz, são realizar operações matemáticas (integração, diferenciação, soma, multiplicação/amplificação, etc.), quando operando na \emph{região linear} (região ativa). Na \emph{região de saturação}, este dispositivo pode ser utilizado como comparador, gerador de onda quadrada, dente de serra, filtros, osciladores, etc.
    
    Neste experimento será estudado o comportamento de circuitos amplificadores operacionais, desde uma configuração simples, como a da Figura \ref{fig:circ2} até uma aplicação mais complexa (acoplada ao \emph{push-pull}) como visto na Figura \ref{fig:circ4}.