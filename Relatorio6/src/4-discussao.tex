\section{Discussão} \label{sec:discussao}
    %%%%%%%%%%%%%%%%%%%%%%%
    %   ITENS 1.1 E 1.2   %
    %%%%%%%%%%%%%%%%%%%%%%%
    Em um primeiro momento buscou-se analisar a linearidade do \emph{push pull} da Figura \ref{fig:circ1}. Aplicando um sinal de entrada triangular de $500mV$ com uma frequência de $1KHz$, percebe-se, como pode ser visto na Figura \ref{fig:foto1}, que essa relação não existe, já que efeitos secundários tais como efeitos de saturação, temperatura, etc não são desprezados. Então buscou-se obter a distorção harmônica do mesmo. Como explicado em \cite{Sedra2004}, o conceito da \emph{transformada de Laplace}, permite que se escreva qualquer onda como uma soma de ondas senoidais. Para a atividade do laboratório aplicou-se uma onda senoidal de $500mV$ e $1KHz$, uma vez que o sinal de entrada foi uma senoide, o espectro da onda é representado por apenas um pico, com amplitude fundamental, como observado na \ref{fig:foto2}, foi de $480mV$. Desse modo, aplicando a Equação \ref{eq:dht}, a distorção foi de $480mV$.
    
    %%%%%%%%%%%%%%%%
    %   ITEM 1.3   %
    %%%%%%%%%%%%%%%%
    Para o circuito da Figura \ref{fig:circ2} dois novos componentes foram introduzidos, o amplificador operacional e o potenciômetro. O primeiro é internamente descrito como na Figura \ref{fig:ampOp}, por isso é importante alimentá-lo com uma tensão de entrada contínua a fim de polarizá-lo, já o segundo, é um tipo especial de resistor de três terminais cuja resistência pode ser ajustada por meio mecânico, girando ou deslizando um eixo móvel, formando assim um divisor de tensão ajustável. Uma vez entendido o funcionamento dos componentes, foi possível analisar o circuito. É interessante notar que este funciona adicionando uma espécie de \emph{offset} de nível \emph{DC} ao sinal de entrada($V_{in}$), como podemos ver na \ref{fig:foto3}. O \emph{offset} adicionado é provido pelo sinal de saída do divisor de tensão do potênciometro e, uma vez que este é ajustável, o \emph{offset} também será. Essa mudança na tensão do potênciometro permite que se observe um comportamento interessante do \emph{amp op} que é o \emph{range} de tensões de saída que ele suporta, isto é, uma vez que o mesmo estava polarizado com $V_{CC}=12V$ e $V_{EE}=0V$, nenhum sinal menor que $0V$ ou maior que $12V$ será emitido na saída, portanto, para se obter um melhor aproveitamento do circuito é preciso encontrar uma tensão no potenciômetro que acrescente um \emph{offset} que seja totalmente aproveitado, isto é, que não seja limitado em nenhum dos limites.
    
    %%%%%%%%%%%%%%%%
    %   ITEM 1.4   %
    %%%%%%%%%%%%%%%%
    Após entender o comportamento do \emph{amp op} com entrada no \emph{inversor}, acoplamos o circuito da Figura \ref{fig:circ2} a um amplificador com realimentação, montando o circuito da Figura \ref{fig:circ3}, neste momento, obtêm-se uma montagem que irá aplicar um \emph{offset} ao sinal de entrada e depois amplificá-lo. Um detalhe interessante é que, visto que o ganho do amplificador operacional é muito alto, usa-se a técnica da realimentação (que consiste em fazer com que o sinal de saída passe por um divisor de tensão e então se torne um sinal de entrada) para amenizar a amplificação total do circuito. Assim como no tópico anterior, o amplificador em questão também possui o sinal de saída limitado (vide Figura \ref{fig:foto6}), portanto, é preciso encontrar o ponto de excursão máxima do amplificador (observado na Figura \ref{fig:foto7}, para que se aproveite, ao máximo, o sinal de entrada, sem distorções ou perda de partes do sinal. Outra característica interessante dos amplificadores operacionais é que eles podem funcionar como um filtro (assim como os estudados com mais detalhe no primeiro experimento, que pode ser visto em \cite{ref:exp1}). A fim de entender esse comportamento variou-se a frequência do sinal de entrada em um \emph{range} que foi de $500Hz$ até $2MHz$, nessa situação foi possível concluir que o \emph{amp op} funciona como um filtro \emph{Passa-Baixa}, como o da Figura \ref{fig:passaBaixa}, tal comportamento acontece pois existe um atraso interno, ou seja, um tempo pequeno até a entrada sair pelo terminal de saída do \emph{amp op}, logo, se a onda é muito rápida (frequência muito alta), esse processamento não ocorre há tempo, e então o \emph{feedback} não existe e o amplificador operacional deixa de funcionar. Tal comportamento pode ser também observado nas Figuras \ref{fig:foto9} e \ref{fig:foto10}.

    %%%%%%%%%%%%%%%%%%%
    %   ITENS 2 E 3   % ==> foi o que a gente não fez
    %%%%%%%%%%%%%%%%%%%
    Ao decorrer do experimento deveriam ser montados e estudados os circuitos mostrados pelas Figuras \ref{fig:circ4} e \ref{fig:circ5}, no entanto, ao tentar acoplar o circuito \emph{push pull} ao circuito da Figura \ref{fig:circ3} esqueceu-se de conectar a fonte de alimentação constante ao \emph{push pull}, nesse momento também aumentou-se consideravelmente a fonte variável $V_{in}$ o que provavelmente ocasionou a queima do transistor \emph{NPN} e, por conseguinte, o não funcionamento do circuito, por essa razão não foi possível realizar o experimento. A fim de conseguir entender e estudar as discussões relacionadas o grupo simulou o comportamento do circuito em um \emph{software} online, como pode ser visto em \cite{ref:simu}. Para o circuito em questão seria necessário observar que o sinal de saída sofre a ação de um filtro \emph{Passa-Alta} e deveria mostrar um comportamento como o da Figura \ref{fig:passaAlta}.
